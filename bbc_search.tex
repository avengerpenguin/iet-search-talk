\documentclass{beamer} 

\usetheme{Pittsburgh}
\usecolortheme{beaver}

\usepackage{dot2texi}
\usepackage{tikz}
\usetikzlibrary{shapes,arrows}
\usepackage{bookmark}
\usepackage{graphicx}

\title{Web Search and the BBC}
\author{Ross Fenning}
\institute{Senior Software Engineer\\Homepage, Search and Navigation\\Future Media\\BBC}
\date{23 May, 2013}

\begin{document}

\begin{frame}[plain]
  \titlepage
\end{frame}

\begin{frame}
  \frametitle{What is BBC Search?}
  \framesubtitle{i.e. What do I work on?}
  \includegraphics[width=\linewidth]{homepage.png}
\end{frame}
\note{
  I work in the team responsible for BBC's online search facility. The
  first point of entry is usually a the search box on the top right
  as shown. This then takes you to a search results page.
}
\begin{frame}
  \frametitle{BBC Search results}
  \includegraphics[width=\linewidth]{results.png}
\end{frame}
\note{
  This is the page you get after having done a search if you started
  on the home page as shown previously. You can see the first
  results are chosen by editors as the best results you are likely
  looking for.
}

\begin{frame}
  \frametitle{BBC iPlayer Search results}
  \includegraphics[width=\linewidth]{iplayer.png}
\end{frame}
\note{
  Some parts of the BBC website have a specialised, ``scoped'' search
  that only searches amongst content for that part of the site. Here,
  we see the result of having clicked the ``iPlayer'' navigation link
  and then used the search box.

  This is based on the assumption that if someone is already on the
  iplayer pages, then any searches they do will be for only TV and
  Radio programmes that are currently available to watch or listen to.
}

\begin{frame}
  \frametitle{CBBC Search results}
  \includegraphics[width=\linewidth]{cbbc.png}
\end{frame}
\note{
  Here, we have a search page that only shows pages and content suitable
  for children under 13. The search scoping is now not just for the
  convenience of only showing pages for the section we're in, but also
  providing a safe, appropriate subsection of the wider BBC website
  for children where they are only navigating, searching and viewing
  things that are known to be suitable.
}

\begin{frame}
  \frametitle{What's happening with BBC search?}
  \begin{itemize}
    \pause \item New team
    \begin{itemize}
    \pause \item Based in BBC North
    \end{itemize}
    \pause \item Improve architecture
    \pause \item Improve user experience
  \end{itemize}
\end{frame}
\note{
  The search facility has run without a lot of changes for a few years now.
  It has recently been identified as one of the major online ``products'' as
  part of the BBC online service and now has a new, prominent team in
  MediaCityUK, Salford. We are initially looking to change the commercial
  provider of the core search engine itself and then do some improvements
  that won't be too visible to the audience right away. Hopefully once
  we've cut out old parts that are no longer in use and generally improve
  the architecture, it will make it easier to try a lot more visible
  improvements to the user experience.
}

\begin{frame}
  \frametitle{Outline of this talk}
  \begin{enumerate}
    \pause \item What is web search?
    \pause \item Search techniques
    \pause \item Search design patterns
    \pause \item Why does BBC have its own search?
    \pause \item What does the BBC have online?
    \pause \item What does BBC Search look like as software system?
    \pause \item Model-View-Controller (MVC)
    \pause \item Service-Orientated Architecture (SOA)
    \pause \item Event-Driven Architecture (EDA)
    \pause \item How can we index BBC content?
    \pause \item How do we search BBC content?
    \pause \item How can we keep improving the experience?
  \end{enumerate}
\end{frame}
\note{
  Today, I will talk about what web search is academically and generally
  then introduce the ``why'' and the ``what'' of BBC search
  Then we can look at some specific software engineering principles
  relevant to such a large software project and then move on
  to some of the challenges faced by doing a search for the whole of
  BBC online.
}

% Section 1

\begin{frame}
  \frametitle{What is the web?}
  \begin{enumerate}
    \pause \item Collection of pages/documents
    \pause \item Each has a unique address, e.g. http://en.wikipedia.org/wiki/World\_Wide\_Web
    \pause \item Those addresses are known as \emph{Unique Resource Locators} (URLs)
    \pause \item Pages/documents contain \emph{hyperlinks} to other pages/documents
    \pause \item We now have images, videos, games, etc. and not just pages and documents
  \end{enumerate}
\end{frame}
\note{
  Firstly, it might be useful to outline a precise and distilled
  definition of what the World Wide Web (or WWW or just ``web'')
  actually is.

  The Web is essentially the collection of all pages we can get to
  with a web browser, each with a unique address, known as a
  Unique Resource Locator or URL. Where the web metaphor
  comes in is that these pages can link to each other by embedding
  each other's URLs and thus pages can reference each other easily.

  Of course, it's not just pages that make up this web, but images
  videos, games, machine-readable information, etc. For a web
  search intended for use by people -- as it the case with Google
  primarily or the BBC Search -- we can focus for now on pages
  (and to some extent videos, games, etc. as people are interested
  in watching and playing those, but they bring their own challenges).
}

\begin{frame}[fragile]
  \frametitle{What is web search?}
  \begin{center}
    \begin{dot2tex}[dot,mathmode,scale=0.8]
      digraph G {
        rankdir=LR;
        node [shape="circle"];
        Crawling -> Indexing -> Querying;
      }
    \end{dot2tex}
  \end{center}
  \begin{enumerate}
    \item Crawl the web to find new pages
    \item Index pages to make it easier to search
    \item Searching/querying the page index
  \end{enumerate}
\end{frame}
\note {
  The process involved in putting together any search engine such
  as Google can essentially be broken down into three phases:
  1) web crawling or otherwise acquiring a list of pages/items that you
  want the search to be able to find; 2) organising those pages/items
  into an index that is easily and quickly searchable; and 3) performing
  searches or queries against that index to present a list of
  pages that are likely of interest to the person who entered a search.

  We can aliken this to creating even a paper index of books within
  a library without even involving a computer-based system. First,
  we must acquire a list of books the library has, which we can do
  by walking along all the shelves and taking note of what we have.
  Then we must take this list of books and organise them in some way.
  For example, we could reorder them into a system such as Dewey Decimal
  where they are grouped by genre or category. We could keep a list
  of which floor of our library has which categories and which books
  we have in each category. We could even
  write some extra keywords on our list next to each book so make it
  easier to see at a glance precisely what each book is about.
  Then, finally we have a visitor who enters the library, tells us
  they are looking for books on, say, psychology and we can refer
  to our list, perhaps ask them to be more specific about their
  query (child psychology, perhaps) and then we can point them
  to a recommended floor, shelf number and possibly a list of book
  titles they might be interested in.
}

\begin{frame}
  \frametitle{Web Crawlers}
  \begin{enumerate}
    \item Visit a page already known about
    \item Add that page to the search index
    \item Find all the links in that page
    \item Store all those links and visit them as well
  \end{enumerate}
\end{frame}
\note {
  
}

\begin{frame}
  \frametitle{Extracting Content}
  Crawlers/Spiders, Last-Modified, Extracting content
\end{frame}

\begin{frame}
  \frametitle{Indexing}
  \begin{itemize}
  \item Inverted Index
  \item Stemming? Tokenisation? Lemmatisation?
  \end{itemize}
\end{frame}

\begin{frame}
  \frametitle{Searching the index}
\end{frame}

% Section 2

\begin{frame}
  \frametitle{Scoring}
\end{frame}

\begin{frame}
  \frametitle{Google Pagerank}
\end{frame}

\begin{frame}
  \frametitle{Standard Search Interface}
\end{frame}

\begin{frame}
  \frametitle{Autocomplete}
\end{frame}

\begin{frame}
  \frametitle{Autosuggest}
\end{frame}

\begin{frame}
  \frametitle{Spell Check}
\end{frame}

\begin{frame}
  \frametitle{Facets}
\end{frame}

\begin{frame}
  \frametitle{Localisation}
\end{frame}

\begin{frame}
  \frametitle{Advanced Search}
\end{frame}

\begin{frame}
  \frametitle{Similar}
  \framesubtitle{``More like this''}
\end{frame}

\begin{frame}
  \frametitle{Why BBC Search?}
  \framesubtitle{Isn't Google good enough?}
\end{frame}

\begin{frame}
  \frametitle{Understanding BBC TV Programmes}
\end{frame}

\begin{frame}
  \frametitle{Understanding BBC Radio Programmes}
\end{frame}

\begin{frame}
  \frametitle{Linking News and other content by topic}
\end{frame}

\begin{frame}
  \frametitle{Connecting iPlayer with Learning}
\end{frame}

\begin{frame}
  \frametitle{Presenting regional and local information}
\end{frame}

\begin{frame}
  \frametitle{Supporting domestic languages}
\end{frame}

\begin{frame}
  \frametitle{What does BBC Search look like?}
  High-level diagram
\end{frame}

\begin{frame}
  \frametitle{Model-View-Controller (MVC)}
\end{frame}

\begin{frame}
  \frametitle{The problem of too many (or too few) databases}
  Too many databases, single point of failure with one database (decoupling)
\end{frame}

\begin{frame}
  \frametitle{``Service Layer''}
  \framesubtitle{Decouples the web application from the database}
  Decoupling, no need for database
\end{frame}

\begin{frame}
  \frametitle{Providing APIs for other service and applications}
  \framesubtitle{Application Programming Interface}
\end{frame}

\begin{frame}
  \frametitle{Service Orientated Architecture}
  \framesubtitle{Modularity and distributed computing}
\end{frame}

\begin{frame}
  \frametitle{Problems when service interfaces change}
\end{frame}

\begin{frame}
  \frametitle{Information feeds have to be polled}
  \framesubtitle{Or the information producer needs a list of consumers to notify}
\end{frame}

\begin{frame}
  \frametitle{Events}
\end{frame}

\begin{frame}
  \frametitle{Event-Driven Architecture}
\end{frame}

\begin{frame}
  \frametitle{What does BBC Search service layer look like?}
\end{frame}

\begin{frame}
  \frametitle{Different data stores or Content Management Systems (CMS)}
\end{frame}

\begin{frame}
  \frametitle{Indexing programmes information}
  \framesubtitle{Structured data}
\end{frame}

\begin{frame}
  \frametitle{How do we know if a programme is available on iPlayer?}
  \framesubtitle{7-day window}
\end{frame}

\begin{frame}
  \frametitle{Indexing News and Sport stories}
  \framesubtitle{Rapidly-published, text content}
\end{frame}

\begin{frame}
  \frametitle{Indexing Children's games, Bitesize, etc.}
  \frametitle{Content vs. pages}
\end{frame}

\begin{frame}
  \frametitle{Web Crawling}
  \framesubtitle{Older pages, mothballed sites, etc.}
\end{frame}

\begin{frame}
  \frametitle{Items with no tags}
  \framesubtitle{Can we generate them?}
\end{frame}

\begin{frame}
  \frametitle{How do we search it all?}
\end{frame}

\begin{frame}
  \frametitle{Searching different types of content}
  \framesubtitle{Comparing apples and oranges}
\end{frame}

\begin{frame}
  \frametitle{Some things have lots of text}
  \framesubtitle{Some things have no text}
\end{frame}

\begin{frame}
  \frametitle{What categories and facets can we offer?}
\end{frame}

\begin{frame}
  \frametitle{How can we keep improving the experience?}
\end{frame}

\begin{frame}
  \frametitle{Analytics}
\end{frame}

\begin{frame}
  \frametitle{Popular searches}
\end{frame}

\begin{frame}
  \frametitle{TV/Radio Broadcasts}
\end{frame}

\begin{frame}
  \frametitle{Responding to major events}
  Riots, snow
\end{frame}

\begin{frame}
  \frametitle{``Obvious'' Ordering of things}
  Manchester United, Man City, Manchester
\end{frame}

\begin{frame}
  \frametitle{Awareness of time of day...}
  \framesubtitle{... or day of the week}
\end{frame}

\begin{frame}
  \frametitle{Location Awareness}
\end{frame}

\begin{frame}
  \frametitle{User Testing}
\end{frame}

\begin{frame}
  \frametitle{Social Media}
  Trends on Twitter, shares on Facebook
\end{frame}

\begin{frame}
  \frametitle{The Future}
  Solve all the problems raised
\end{frame}

\end{document}

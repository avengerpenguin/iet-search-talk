\documentclass{beamer} 

\usetheme{Pittsburgh}
\usecolortheme{beaver}

\usepackage{dot2texi}
\usepackage{tikz}
\usetikzlibrary{shapes,arrows}
\usepackage{bookmark}
\usepackage{graphicx}

\title{Web Search and the BBC}
\author{Ross Fenning}
\institute{Senior Software Engineer\\Homepage, Search and Navigation\\Future Media\\BBC}
\date{23 May, 2013}

\begin{document}

\begin{frame}[plain]
  \titlepage
\end{frame}

\begin{frame}
  \frametitle{What is BBC Search?}
  \framesubtitle{i.e. What do I work on?}
  \includegraphics[width=\linewidth]{homepage.png}
\end{frame}
\note{
  I work in the team responsible for BBC's online search facility. The
  first point of entry is usually a the search box on the top right
  as shown. This then takes you to a search results page.
}

\begin{frame}
  \frametitle{BBC Search results}
  \includegraphics[width=\linewidth]{results.png}
\end{frame}
\note{
  This is the page you get after having done a search if you started
  on the home page as shown previously. You can see the first
  results are chosen by editors as the best results you are likely
  looking for.
}

\begin{frame}
  \frametitle{BBC iPlayer Search results}
  \includegraphics[width=\linewidth]{iplayer.png}
\end{frame}
\note{
  Some parts of the BBC website have a specialised, ``scoped'' search
  that only searches amongst content for that part of the site. Here,
  we see the result of having clicked the ``iPlayer'' navigation link
  and then used the search box.

  This is based on the assumption that if someone is already on the
  iplayer pages, then any searches they do will be for only TV and
  Radio programmes that are currently available to watch or listen to.
}

\begin{frame}
  \frametitle{CBBC Search results}
  \includegraphics[width=\linewidth]{cbbc.png}
\end{frame}
\note{
  Here, we have a search page that only shows pages and content suitable
  for children under 13. The search scoping is now not just for the
  convenience of only showing pages for the section we're in, but also
  providing a safe, appropriate subsection of the wider BBC website
  for children where they are only navigating, searching and viewing
  things that are known to be suitable.
}

\begin{frame}
  \frametitle{What's happening with BBC search?}
  \begin{itemize}
    \pause \item New team
    \begin{itemize}
    \pause \item Based in BBC North
    \end{itemize}
    \pause \item Improve architecture
    \pause \item Improve user experience
  \end{itemize}
\end{frame}
\note{
  The search facility has run without a lot of changes for a few years now.
  It has recently been identified as one of the major online ``products'' as
  part of the BBC online service and now has a new, prominent team in
  MediaCityUK, Salford. We are initially looking to change the commercial
  provider of the core search engine itself and then do some improvements
  that won't be too visible to the audience right away. Hopefully once
  we've cut out old parts that are no longer in use and generally improve
  the architecture, it will make it easier to try a lot more visible
  improvements to the user experience.
}

\begin{frame}
  \frametitle{Outline of this talk}
  \begin{enumerate}
    \pause \item What is web search?
    \pause \item Search techniques
    \pause \item Search design patterns
    \pause \item Why does BBC have its own search?
    \pause \item What does the BBC have online?
    \pause \item What does BBC Search look like as software system?
    \pause \item Model-View-Controller (MVC)
    \pause \item Service-Orientated Architecture (SOA)
    \pause \item Event-Driven Architecture (EDA)
    \pause \item How can we index BBC content?
    \pause \item How do we search BBC content?
    \pause \item How can we keep improving the experience?
  \end{enumerate}
\end{frame}
\note{
  Today, I will talk about what web search is academically and generally
  then introduce the ``why'' and the ``what'' of BBC search
  Then we can look at some specific software engineering principles
  relevant to such a large software project and then move on
  to some of the challenges faced by doing a search for the whole of
  BBC online.
}

% Section 1

\begin{frame}
  \frametitle{What is the web?}
  \begin{enumerate}
    \pause \item Collection of pages/documents
    \pause \item Each has a unique address, e.g. http://en.wikipedia.org/wiki/World\_Wide\_Web
    \pause \item Those addresses are known as \emph{Uniform Resource Locators} (URLs)
    \pause \item Pages/documents contain \emph{hyperlinks} to other pages/documents
    \pause \item We now have images, videos, games, etc. and not just pages and documents
  \end{enumerate}
\end{frame}
\note{
  Firstly, it might be useful to outline a precise and distilled
  definition of what the World Wide Web (or WWW or just ``web'')
  actually is.

  The Web is essentially the collection of all pages we can get to
  with a web browser, each with a unique address, known as a
  Uniform Resource Locator or URL. Where the web metaphor
  comes in is that these pages can link to each other by embedding
  each other's URLs and thus pages can reference each other easily.

  Of course, it's not just pages that make up this web, but images
  videos, games, machine-readable information, etc. For a web
  search intended for use by people -- as it the case with Google
  primarily or the BBC Search -- we can focus for now on pages
  (and to some extent videos, games, etc. as people are interested
  in watching and playing those, but they bring their own challenges).
}

\begin{frame}[fragile]
  \frametitle{What is web search?}
  \begin{center}
    \begin{dot2tex}[dot,mathmode,scale=0.8]
      digraph G {
        rankdir=LR;
        node [shape="circle"];
        Crawling -> Indexing -> Querying;
      }
    \end{dot2tex}
  \end{center}
  \begin{enumerate}
    \item Crawl the web to find new pages
    \item Index pages to make it easier to search
    \item Searching/querying the page index
  \end{enumerate}
\end{frame}
\note {
  The process involved in putting together any search engine such
  as Google can essentially be broken down into three phases:
  1) web crawling or otherwise acquiring a list of pages/items that you
  want the search to be able to find; 2) organising those pages/items
  into an index that is easily and quickly searchable; and 3) performing
  searches or queries against that index to present a list of
  pages that are likely of interest to the person who entered a search.

  We can aliken this to creating even a paper index of books within
  a library without even involving a computer-based system. First,
  we must acquire a list of books the library has, which we can do
  by walking along all the shelves and taking note of what we have.
  Then we must take this list of books and organise them in some way.
  For example, we could reorder them into a system such as Dewey Decimal
  where they are grouped by genre or category. We could keep a list
  of which floor of our library has which categories and which books
  we have in each category. We could even
  write some extra keywords on our list next to each book so make it
  easier to see at a glance precisely what each book is about.
  Then, finally we have a visitor who enters the library, tells us
  they are looking for books on, say, psychology and we can refer
  to our list, perhaps ask them to be more specific about their
  query (child psychology, perhaps) and then we can point them
  to a recommended floor, shelf number and possibly a list of book
  titles they might be interested in.
}

\begin{frame}
  \frametitle{Web Crawlers}
  \begin{enumerate}
    \item Visit a page already known about
    \item Add that page to the search index
    \item Find all the links in that page
    \item Store all those links and visit them as well
  \end{enumerate}
\end{frame}
\note {
  A web search such as Google can only know about pages people
  might be looking for by crawling as many pages on the web that
  can be found. This is the equivalent of taking stock and writing
  down all the books on the library shelf, except in this case
  we have a whole web of pages and content that is effectively boundless --
  a stark contrast to a library of very finite size.

  We can start a crawling process by visiting a page we know about -- let's
  say the front page of Wikipedia. We can note the content of that page
  and index it, but the page itself will contain links to many featured articles.
  We can keep a note of all those links, follow them to the articles themselves
  and repeat the crawl process using all the links on those pages.
}

\begin{frame}
  \frametitle{Extracting Content}
  \includegraphics[width=\linewidth]{wikipedia.png}
\end{frame}
\note{
  So, we have a list of pages, like the Wikipedia article shown. Before we
  can create our index, there is at least one important challenge to overcome
  first. Looking at the Wikipedia article shown, there's text that doesn't
  actually form part of the article such as the navigation menu down the side.

  We don't want to suggest this article when someone searches for ``current events''
  or ``random article'', so we need to find a way to identify what is part
  of the actual article and what is not. It's probably clear to us as humans, but
  how do we make a machine understand that distinction? It's less obvious to a
  computer and different websites have different layouts, so this can be quite
  an interesting challenge.
}

\begin{frame}
  \frametitle{Indexing}
  \framesubtitle{e.g. Inverted Index}
  \begin{center}
    \begin{tabular}{|l|l|}
      \hline
      \bf{Documents}   & \bf{Words}             \\ \hline
      Document 1 & three,blind,mice  \\ \hline
      Document 2 & three,little,pigs \\ \hline
      Document 3 & pigs,in,blankets  \\ \hline
    \end{tabular}

    $\downarrow$

    \begin{tabular}{|l|l|}
      \hline
      \bf{Words}     & \bf{Documents}              \\ \hline
      three    & Document 1,Document 2\\ \hline
      pigs     & Document 2,Document 3 \\ \hline
      blind    & Document 1             \\ \hline
      mice     & Document 1             \\ \hline
      little   & Document 2             \\ \hline
      in       & Document 3             \\ \hline
      blankets & Document 3             \\ \hline
    \end{tabular}

  \end{center}
\end{frame}
\note{
  After having crawled or otherwise collected up a collection of documents or
  pages that we might want to present in search results, we need to create
  a search index that will enable accurate and quick queries to be made against
  it. Clearly, reading through every article for keywords entered by a user
  every time a new query in is an inefficient way to work. The process of indexing
  arranges all your documents and pages in such a way as to make searches easier,
  much like the Dewey Decimal categorisation in our library example earlier.

  A very basic form of search index is the \emph{inverted index}. Here, we create
  a lookup table where each word points to the documents that contain that word.
  In the example shown, if a user types in the word ``pigs'', we can point them
  at the second and third document. Note how we can do that with one lookup and
  we are not required to read through each document looking for the word ``pigs'',
  which would be much more time consuming on large documents.

  There are many more sophisticated structures and arrangements that can
  bring other benefits, but this is already a much faster way to find documents
  that match a keyword. However, if we were to add a document that contains
  the word ``mouse'', should that be put in the same row along with the first
  document that contains mouse? This leads us onto the linguistic techniques we
  need to consider when building a search index.
}

\begin{frame}
  \frametitle{Linguistics Techniques}
  \begin{itemize}
    \pause \item Tokenisation
    \begin{itemize}
      \pause \item Breaking sentences up into words or \emph{tokens}
      \pause \item English generally uses spaces, but Chinese does not
      \pause \item \emph{o'clock}? \emph{twenty-one}? \emph{:-)}? \emph{Robin Hood}?
    \end{itemize}
    \pause \item Stemming
    \begin{itemize}
      \pause \item Remove suffixes so that, e.g.  swim and swimming are the same
      \pause \item What's the stem of argue and arguing? Dry and dries?
      \pause \item Overstemming: \emph{Music}, \emph{Musical} and \emph{Musician}
    \end{itemize}
    \pause \item Lemmatisation
    \begin{itemize}
      \pause \item Swim, Swam Swum
      \pause \item Think, Thought
      \pause \item Good, Better, Best
    \end{itemize}
    \pause \item All of this only applies to \emph{English}
    \pause \item Other languages need their own rules
  \end{itemize}
\end{frame}
\note {
  In the previous slide, we assumed we had already extracted all the words out
  of a document, but even that process might not be as simple as it would seem.
  The process for turning paragraphs of text into individual words or \emph{tokens}
  as the more exact term should be is known as \emph{tokenisation}.

  For English, we can assume a space breaks text into words, but not all languages
  behave this way. Chinese has no boundaries between words. Even in English, we
  sometimes have things like apostrophes or hyphens. Perhaps we might want to
  identify very specific tokens such as smiley faces or maybe we want to treat
  names like \emph{Robin Hood} as a single token since the individual words
  \emph{robin} and \emph{hood} carry very different meanings when encountered
  individually.

  Once we have our words, we probably don't want swim and swimming to be seen
  differently as they are essentially the same word. This is know as \emph{stemming}.
  Stemming gets more complex that just removing letters when you have things like
  \emph{dry} and \emph{dries} where you have to change letters as well as remove
  to get the common stem. Then the problem of \emph{overstemming} can occur when
  removing what looks like simple suffices actually changes the word. Here, we can
  see that \emph{musical} is fine to lose the suffix \emph{-al} where
  it's the adjective from the noun \emph{music}, but it might be incorrect when a document
  is taking about the noun \emph{musical} as in a stage production since those two
  concepts aren't exactly the same thing. It gets worse still if our stemming
  process also decided to reduce \emph{musician} down to the same as music as that
  is, again, a different concept.

  Lemmatisation overlaps somewhat with stemming in that stemming could be seen
  as a na\"ive approach to lemmatisation. Our goal here is not just to reduce
  a word to some root stem before using it for our index, but also to find
  all inflections or variants and reduce them to their \emph{lemma}. So \emph{swimming}
  was previously stemmed to swim, but we also want to recognise that \emph{swam}
  and \emph{swum} could probably be converted to \emph{swim} as well before indexing.
  We have other examples such as think, thought, catch, caught. It's not just verbs
  either as \emph{better} and \emph{best} could be recognised as variants of
  \emph{good}.

  The main difference between stemming and lemmatisation is that stemming happens
  without any real context of the word and can make more mistakes, but is a lot faster.
  Lemmatisation usually takes into account the meaning of the word in the sentence
  and is less likely to confuse things like \emph{musical} the adjective and
  \emph{musical} the noun for a stage production.

  All of this is fairly technical linguistics terminology, but the easiest way to think of
  it all is that the lemma of a word is the form you would look up in a dictionary. So,
  if you wanted to know the meaning of the word ``caught'' you are likely to look
  up ``catch'' and wouldn't really expect there to be an entry for caught.

  The result of this if we look back at our inverted index example before:
}

\begin{frame}
  \frametitle{Indexing}
  \framesubtitle{e.g. Inverted Index}
  \begin{center}
    \begin{tabular}{|l|l|}
      \hline
      \bf{Words}     & \bf{Documents}              \\ \hline
      three    & Document 1,Document 2\\ \hline
      \bf{pig}     & Document 2,Document 3 \\ \hline
      blind    & Document 1             \\ \hline
      \bf{mouse}     & Document 1             \\ \hline
      little   & Document 2             \\ \hline
      in       & Document 3             \\ \hline
      \bf{blanket} & Document 3             \\ \hline
    \end{tabular}

  \end{center}
\end{frame}
\note{
  We see that our index now prefers the singular forms pig, mouse and blanket. So,
  if we were then to add a document with the word ``mouse'' it would be put in the same
  entry as ``three blind mice'' and a someone querying our search index could
  type ``mouse'' or ``mice'' and it would find either entry. This is clearly
  better for users and you can try this out on Google by searching ``mice'' yourself and
  you will likely get plenty of results with ``mouse'' in the title.
}

\begin{frame}
  \frametitle{Searching the index}
  \begin{itemize}
    \pause \item User enters some keywords
    \pause \item Look up documents/pages that contain keywords
    \pause \item Present a list of results
    \pause \item Should results contain all or just some of the keywords?
    \pause \item What order should results be in?
    \pause \item What other information can be used?
  \end{itemize}
\end{frame}
\note{
  The third and final stage in our search system is the actual searching
  of the index itself. We've already covered this a bit since our
  inverted index made this quite easy with one lookup.

  We start by taking the keywords the user typed in and search our index
  for all documents containing those words and then present a list of
  those documents as candidates for what the user was probably seeking.

  There are many more intricate techniques to do with what to do
  if the document only contains some of the words, how best to order
  the results and Google itself takes into account a lot of contextual
  information such as time and where in the world you are.

  We can look briefly at how we decide which are the better matches so
  we can order the results and also how Google deals with the issue
  of wanting to prefer trustworthy sources for a topic.
}

% Section 2

\begin{frame}
  \frametitle{Scoring}
\end{frame}

\begin{frame}
  \frametitle{Google Pagerank}
\end{frame}
\note {
  One disadvantage of
  the World Wide Web is the ease of creating pages allows for a sea
  of non-trustworthy, inaccurate, malicious and unhelpful sites that
  can make finding a reliable and suitable site a difficult endeavour.
}

\begin{frame}
  \frametitle{Standard Search Interface}
\end{frame}

\begin{frame}
  \frametitle{Autocomplete}
\end{frame}

\begin{frame}
  \frametitle{Autosuggest}
\end{frame}

\begin{frame}
  \frametitle{Spell Check}
\end{frame}

\begin{frame}
  \frametitle{Facets}
\end{frame}

\begin{frame}
  \frametitle{Localisation}
\end{frame}

\begin{frame}
  \frametitle{Advanced Search}
\end{frame}

\begin{frame}
  \frametitle{Similar}
  \framesubtitle{``More like this''}
\end{frame}

\begin{frame}
  \frametitle{Why BBC Search?}
  \framesubtitle{Isn't Google good enough?}
\end{frame}

\begin{frame}
  \frametitle{Understanding BBC TV Programmes}
\end{frame}

\begin{frame}
  \frametitle{Understanding BBC Radio Programmes}
\end{frame}

\begin{frame}
  \frametitle{Linking News and other content by topic}
\end{frame}

\begin{frame}
  \frametitle{Connecting iPlayer with Learning}
\end{frame}

\begin{frame}
  \frametitle{Presenting regional and local information}
\end{frame}

\begin{frame}
  \frametitle{Supporting domestic languages}
\end{frame}

\begin{frame}
  \frametitle{What does BBC Search look like?}
  High-level diagram
\end{frame}

\begin{frame}
  \frametitle{Model-View-Controller (MVC)}
\end{frame}

\begin{frame}
  \frametitle{The problem of too many (or too few) databases}
  Too many databases, single point of failure with one database (decoupling)
\end{frame}

\begin{frame}
  \frametitle{``Service Layer''}
  \framesubtitle{Decouples the web application from the database}
  Decoupling, no need for database
\end{frame}

\begin{frame}
  \frametitle{Providing APIs for other service and applications}
  \framesubtitle{Application Programming Interface}
\end{frame}

\begin{frame}
  \frametitle{Service Orientated Architecture}
  \framesubtitle{Modularity and distributed computing}
\end{frame}

\begin{frame}
  \frametitle{Problems when service interfaces change}
\end{frame}

\begin{frame}
  \frametitle{Information feeds have to be polled}
  \framesubtitle{Or the information producer needs a list of consumers to notify}
\end{frame}

\begin{frame}
  \frametitle{Events}
\end{frame}

\begin{frame}
  \frametitle{Event-Driven Architecture}
\end{frame}

\begin{frame}
  \frametitle{What does BBC Search service layer look like?}
\end{frame}

\begin{frame}
  \frametitle{Different data stores or Content Management Systems (CMS)}
\end{frame}

\begin{frame}
  \frametitle{Indexing programmes information}
  \framesubtitle{Structured data}
\end{frame}

\begin{frame}
  \frametitle{How do we know if a programme is available on iPlayer?}
  \framesubtitle{7-day window}
\end{frame}

\begin{frame}
  \frametitle{Indexing News and Sport stories}
  \framesubtitle{Rapidly-published, text content}
\end{frame}

\begin{frame}
  \frametitle{Indexing Children's games, Bitesize, etc.}
  \frametitle{Content vs. pages}
\end{frame}

\begin{frame}
  \frametitle{Web Crawling}
  \framesubtitle{Older pages, mothballed sites, etc.}
\end{frame}

\begin{frame}
  \frametitle{Items with no tags}
  \framesubtitle{Can we generate them?}
\end{frame}

\begin{frame}
  \frametitle{How do we search it all?}
\end{frame}

\begin{frame}
  \frametitle{Searching different types of content}
  \framesubtitle{Comparing apples and oranges}
\end{frame}

\begin{frame}
  \frametitle{Some things have lots of text}
  \framesubtitle{Some things have no text}
\end{frame}

\begin{frame}
  \frametitle{What categories and facets can we offer?}
\end{frame}

\begin{frame}
  \frametitle{How can we keep improving the experience?}
\end{frame}

\begin{frame}
  \frametitle{Analytics}
\end{frame}

\begin{frame}
  \frametitle{Popular searches}
\end{frame}

\begin{frame}
  \frametitle{TV/Radio Broadcasts}
\end{frame}

\begin{frame}
  \frametitle{Responding to major events}
  Riots, snow
\end{frame}

\begin{frame}
  \frametitle{``Obvious'' Ordering of things}
  Manchester United, Man City, Manchester
\end{frame}

\begin{frame}
  \frametitle{Awareness of time of day...}
  \framesubtitle{... or day of the week}
\end{frame}

\begin{frame}
  \frametitle{Location Awareness}
\end{frame}

\begin{frame}
  \frametitle{User Testing}
\end{frame}

\begin{frame}
  \frametitle{Social Media}
  Trends on Twitter, shares on Facebook
\end{frame}

\begin{frame}
  \frametitle{The Future}
  Solve all the problems raised
\end{frame}

\end{document}
